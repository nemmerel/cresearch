% BEAMER TEMPLATE — Copy this file to slides/ and rename for each lecture.
% Compile with: pdflatex or latexmk -pdf
\documentclass[aspectratio=169]{beamer}

% Use a simple white theme
\usetheme{default}
\usecolortheme{default}

% Packages
\usepackage{amsmath}
\usepackage{amssymb}
\usepackage{graphicx}

% Code listings
\usepackage{listings}
\lstset{
  basicstyle=\ttfamily\footnotesize,
  breaklines=true,
  columns=fullflexible,
  keepspaces=true,
  backgroundcolor=\color{white},
  frame=none,
  xleftmargin=1em,
}

% Times New Roman font
\usepackage[T1]{fontenc}
\usepackage{newtxtext}
\usepackage{newtxmath}

% Define dark red color
\definecolor{darkred}{RGB}{139,0,0}

% Set title colors to dark red
\setbeamercolor{title}{fg=darkred}
\setbeamercolor{frametitle}{fg=darkred}
\setbeamercolor{author}{fg=black}
\setbeamercolor{institute}{fg=black}

% Set itemize bullets to dark red
\setbeamercolor{itemize item}{fg=darkred}
\setbeamercolor{itemize subitem}{fg=darkred}
\setbeamercolor{itemize subsubitem}{fg=darkred}

% Set bullet markers to circles
\setbeamertemplate{itemize item}{\small\raise0.5pt\hbox{\textbullet}}
\setbeamertemplate{itemize subitem}{\scriptsize\raise1pt\hbox{\textbullet}}
\setbeamertemplate{itemize subsubitem}{\tiny\raise1.5pt\hbox{\textbullet}}

% Remove navigation symbols
\setbeamertemplate{navigation symbols}{}

% Simple footer
\setbeamertemplate{footline}[frame number]

% Title information
\title{COURSE-NAME Lecture N:\\
LECTURE-TITLE}
\author{YOUR-NAME}
\institute{YOUR-INSTITUTION}
\date{}

\begin{document}

% Title slide
\begin{frame}
\titlepage
\end{frame}

%%% ---------------------------------------------------------------
%%% EXAMPLE SLIDES — Delete these and replace with your content.
%%% Each slide below demonstrates a common pattern.
%%% ---------------------------------------------------------------

% --- Pattern: Bullet points ---
\begin{frame}{Section Title}
\begin{itemize}
\item First main point

\item Second main point
\begin{itemize}
\item Supporting detail
\item Supporting detail
\end{itemize}

\item Third main point
\end{itemize}
\end{frame}

% --- Pattern: Inline and display math ---
\begin{frame}{Definition or Derivation Title}
\begin{itemize}
\item The model is given by
\[
y_t = \phi y_{t-1} + \varepsilon_t
\]

\item Under the assumption that $|\phi| < 1$, we obtain
\begin{align*}
\mathbb{E}[y_t] &= 0 \\
\mathrm{var}[y_t] &= \frac{\sigma^2}{1 - \phi^2}
\end{align*}
\end{itemize}
\end{frame}

% --- Pattern: Theorem or formal statement ---
\begin{frame}{Theorem Title}
\textbf{Theorem.} \textit{Statement of the theorem in italics.}

\vspace{1em}
\begin{itemize}
\item Implication or interpretation

\item Key condition or assumption
\end{itemize}
\end{frame}

% --- Pattern: Figure ---
% \begin{frame}{FIGURE-TITLE}
% \begin{center}
% \includegraphics[width=0.7\textwidth]{figures/FIGURE_NAME.png}
% \end{center}
% \end{frame}

% --- Pattern: Code listing (use [fragile] option) ---
\begin{frame}[fragile]{Code Example Title}
\begin{itemize}
\item Description of what the code does:
\end{itemize}
\begin{lstlisting}
* [Stata / R / Python code here]
clear all
set seed 20260210
set obs 500
\end{lstlisting}
\vspace{0.5em}
\begin{itemize}
\item Interpretation of results
\end{itemize}
\end{frame}

% --- Pattern: Two-part slide with spacer ---
\begin{frame}{Equation + Commentary Title}
\begin{align*}
F &= \frac{(\mathrm{RSS_{full}} - (\mathrm{RSS_1} + \mathrm{RSS_2}))/k}{(\mathrm{RSS_1} + \mathrm{RSS_2})/(n_1 + n_2 - 2k)}
\end{align*}

\vspace{1em}
\begin{itemize}
\item Where $k$ is the parameter description

\item This is known as the test/method name
\end{itemize}
\end{frame}

\end{document}
