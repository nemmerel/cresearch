\documentclass[12pt]{article}

% --- Shorthand commands ---
\newcommand{\be}{\begin{equation}}
\newcommand{\ee}{\end{equation}}

% --- Packages ---
\usepackage{amsmath,amsthm,amssymb}
\usepackage{graphicx}
\usepackage[left=1.6in,right=1.6in,top=1in,bottom=1in]{geometry}
\usepackage{float}
\usepackage{times}
\usepackage{breqn}
\usepackage{multicol}
\usepackage{hyperref}
\usepackage{natbib}
\usepackage{fancyhdr}
\usepackage{titlesec}
\usepackage{enumitem}

% --- Page style ---
\pagestyle{fancy}
\setlength{\headheight}{15pt}
\lhead{}
\rhead{}
%\chead{\sc{PRELIMINARY: DO NOT DISTRIBUTE}}

% --- Section formatting ---
\titleformat*{\section}{\Large\bfseries\sffamily}
\titleformat*{\subsection}{\large\bfseries\sffamily}
\titleformat*{\subsubsection}{\large\bfseries\sffamily}
\titleformat*{\paragraph}{\bfseries\sffamily}
\titleformat*{\subparagraph}{\bfseries\sffamily}

% --- Theorem environments ---
\newtheorem{prop}{Proposition}
\newtheorem{remark}{Remark}
\newtheorem{cor}{Corollary}
\newtheorem{theorem}{Theorem}
\newtheorem{definition}{Definition}


\begin{document}

\title{\sc{Problem Set [NUMBER]: [COURSE]\footnote{\baselineskip=11pt [INSTITUTION].}}}
\author{[AUTHOR NAME]\footnote{\baselineskip=11pt e-mail: [EMAIL]}}
\date{Due: [DUE DATE]}
\maketitle
\thispagestyle{empty}

\vspace{20pt}

\addtocounter{page}{-1}
\thispagestyle{fancy}

\section{Directions}

[Insert directions here. For example: This problem set is due at the beginning of class on the date above. You may work in groups of up to three students, but each student must write up their own solutions. Show all work. Late submissions will not be accepted.]

\section{[Problem 1 Title]}

\begin{enumerate}[label=(\alph*)]
    \item [First part of the problem.]

    \item [Second part of the problem.]

    \item [Third part of the problem.]
\end{enumerate}

\section{[Problem 2 Title]}

[Problem statement here. You can use display math like this:]
\be
    y_i = \alpha + \beta x_i + \varepsilon_i
\ee

\begin{enumerate}[label=(\alph*)]
    \item [First part of the problem.]

    \item [Second part of the problem.]
\end{enumerate}

\section{[Problem 3 Title]}

[Problem statement here.]

\begin{enumerate}[label=(\alph*)]
    \item [First part of the problem.]

    \item [Second part of the problem.]
\end{enumerate}


\end{document}
